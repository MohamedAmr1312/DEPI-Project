% Options for packages loaded elsewhere
\PassOptionsToPackage{unicode}{hyperref}
\PassOptionsToPackage{hyphens}{url}
%
\documentclass[
]{article}
\usepackage{amsmath,amssymb}
\usepackage{lmodern}
\usepackage{iftex}
\ifPDFTeX
  \usepackage[T1]{fontenc}
  \usepackage[utf8]{inputenc}
  \usepackage{textcomp} % provide euro and other symbols
\else % if luatex or xetex
  \usepackage{unicode-math}
  \defaultfontfeatures{Scale=MatchLowercase}
  \defaultfontfeatures[\rmfamily]{Ligatures=TeX,Scale=1}
\fi
% Use upquote if available, for straight quotes in verbatim environments
\IfFileExists{upquote.sty}{\usepackage{upquote}}{}
\IfFileExists{microtype.sty}{% use microtype if available
  \usepackage[]{microtype}
  \UseMicrotypeSet[protrusion]{basicmath} % disable protrusion for tt fonts
}{}
\makeatletter
\@ifundefined{KOMAClassName}{% if non-KOMA class
  \IfFileExists{parskip.sty}{%
    \usepackage{parskip}
  }{% else
    \setlength{\parindent}{0pt}
    \setlength{\parskip}{6pt plus 2pt minus 1pt}}
}{% if KOMA class
  \KOMAoptions{parskip=half}}
\makeatother
\usepackage{xcolor}
\IfFileExists{xurl.sty}{\usepackage{xurl}}{} % add URL line breaks if available
\IfFileExists{bookmark.sty}{\usepackage{bookmark}}{\usepackage{hyperref}}
\hypersetup{
  hidelinks,
  pdfcreator={LaTeX via pandoc}}
\urlstyle{same} % disable monospaced font for URLs
\usepackage{graphicx}
\makeatletter
\def\maxwidth{\ifdim\Gin@nat@width>\linewidth\linewidth\else\Gin@nat@width\fi}
\def\maxheight{\ifdim\Gin@nat@height>\textheight\textheight\else\Gin@nat@height\fi}
\makeatother
% Scale images if necessary, so that they will not overflow the page
% margins by default, and it is still possible to overwrite the defaults
% using explicit options in \includegraphics[width, height, ...]{}
\setkeys{Gin}{width=\maxwidth,height=\maxheight,keepaspectratio}
% Set default figure placement to htbp
\makeatletter
\def\fps@figure{htbp}
\makeatother
\usepackage[normalem]{ulem}
% Avoid problems with \sout in headers with hyperref
\pdfstringdefDisableCommands{\renewcommand{\sout}{}}
\setlength{\emergencystretch}{3em} % prevent overfull lines
\providecommand{\tightlist}{%
  \setlength{\itemsep}{0pt}\setlength{\parskip}{0pt}}
\setcounter{secnumdepth}{-\maxdimen} % remove section numbering
\ifLuaTeX
  \usepackage{selnolig}  % disable illegal ligatures
\fi

\author{}
\date{}

\begin{document}

\begin{quote}
\textbf{Movie recommendation system using machine learning.}

\textbf{Authors:}
\end{quote}

\textbf{Seif Eldin Hussein Mostafa Ali E-mail address:
seif.hussin11@gmail.com}

\textbf{Khaled Waheed Mohamed Ali E-mail address: waheed33603@gmail.com}

\textbf{github
https://github.com/Seifhussein1/Movie-Recommendation-System.git}

\emph{\textbf{OCTOBER SIX UNIVERSITY}}

\emph{\textbf{FACULTY OF INFORMATION SYSTEMS AND COMPUTER SCIENCE (AI
DEPARTMENT)}}

\hypertarget{a-r-t-i-c-l-e-i-n-f-o}{%
\section{\texorpdfstring{a \textsc{r} \textsc{t} i c \textsc{l} e i n f
o}{a r t i c l e i n f o}}\label{a-r-t-i-c-l-e-i-n-f-o}}

\begin{quote}
\emph{Keywords:}

Cosine similarity

Movie recommendation Naïve Bayes

Sentiment analysis Support vector machine KNN

Bagging

Decision Tree

Gradient Boosting

Perceptron

logistic Regression

Extra Trees Classifier

Random Forest

Ada Boost Classifier
\end{quote}

\hypertarget{a-b-s-t-r-a-c-t}{%
\section{\texorpdfstring{\hfill\break
a b s \textsc{t} \textsc{r} a c
\textsc{t}}{ a b s t r a c t}}\label{a-b-s-t-r-a-c-t}}

In the modern world, where technology is at the forefront of every
industry, there has been an overload of information and data. Thus, a
recommendation system comes in handy to deal with this large volume of
data and filter out the useful information which is fast and relevant to
the user's choice. This paper describes an approach to a movie
recommendation system using Cosine Similarity to recommend similar
movies based on the one chosen by the user. Although the existing
recommendation systems get the job done, it does not justify if the
movie is worth spending time on. To enhance the user experience, this
system performs sentiment analysis on the reviews of the movie chosen
using machine learning. Two of the supervised machine learning
algorithms, Naïve Bayes (NB) Classifier and Support Vector Machine (SVM)
Classifier, are used to increase the accuracy and efficiency. This paper
also gives a comparison between NB and SVM on the basis of parameters
like Accuracy, Precision, Recall, and F1 Score. The accuracy score of
SVM came out to be 98.63\% whereas the accuracy score of NB is 97.33\%.
Thus, SVM outweighs NB and proves to be a better fit for Sentiment
Analysis. Additionally, we employed a diverse set of 10 classifiers to
develop a robust movie recommendation system. Our analysis revealed that
the K-Nearest Neighbors (KNN) classifier achieved the lowest accuracy of
96.749\%, while the AdaBoostClassifier demonstrated the highest accuracy
of 99.061\%. Furthermore, we explored the combined performance of
classifiers by pairing them and evaluating their effectiveness using a
voting ensemble approach. Among the combinations, the KNN and Support
Vector Classifier (SVC) pair exhibited the minimum accuracy of 97.471\%.
Conversely, the AdaBoostClassifier and ExtraTreesClassifier pair
demonstrated the highest accuracy of 99.205\%. Additionally, utilizing
three classifiers, 'Decision Tree', 'ExtraTreesClassifier', and
'AdaBoostClassifier', showcases remarkable performance, with
AdaBoostClassifier achieving a maximum accuracy of 99.422\%. These
findings underscore the significance of ensemble methods in enhancing
the recommendation system's performance.

\hypertarget{introduction}{%
\subsection{Introduction}\label{introduction}}

\begin{quote}
Since its invention, the Internet has grown rapidly and continues to
grow each day. The abundance of information available online, has made
it a strenuous task to access the right information quickly and easily
\protect\hyperlink{_bookmark21}{{[}1{]}}. Fortunately, this problem can
be solved with the help of recommendation systems.

Recommendation systems are used extensively today and have found
applications in multiple industries such as e-commerce, retail,
bank-ing, entertainment etc. These systems collect and auto-analyse the
user data to generate personalised recommendations for the users
\protect\hyperlink{_bookmark22}{{[}2{]}}. The most common approaches to
implement recommendation systems are Content-based Filtering (CBF),
Collaborative Filtering (CF) and Hybrid Filtering
\protect\hyperlink{_bookmark23}{{[}3{]}}. CBF is an approach that is
used to analyse the content of each item and recommend other items that
have similar characteristics. CF addresses some of the limitations of
CBF and provides recommenda- tions by comparing the similarities between
the users and the items. It uses the knowledge of the user's previous
preferences as well as the pref- erences of other similar users to
generate a recommendation. Many rec- ommendation systems are also known
to use the Hybrid-filtering tech- nique combining the features of both
CBF and CF methods \protect\hyperlink{_bookmark24}{{[}4{]}}.

A movie's popularity is based on the type of reviews it gets from the

audience . These reviews are also responsible for affecting the choice of
other users. Users are more likely to choose a movie that was
\end{quote}

preferred by most people rather than a movie that was largely disliked
\protect\hyperlink{_bookmark25}{{[}5{]}}. Analyzing these reviews,
ignoring the reviews that contain misleading information also adds to
the difficulty of decision-making
\protect\hyperlink{_bookmark26}{{[}6{]}}. Sentiment Analysis provides a
solution to this problem.

\begin{quote}
We apply 10 classifiers ' KNN', 'SVC', `Bagging', `Decision
Tree',`GradientBoostingClassifier', `Perceptron' , `logistic Regression'
,`ExtraTreesClassifier' , `Random Forest', `AdaBoostClassifier'

Sentiment Analysis facilitates a way to use NLP (natural language
processing) to extract information from a textual source and classify
the statement or word or document as positive or negative. It is very
useful to understand the opinion of the author and indicate the user
experience. Opinion mining uses the concepts of data mining to extract
and clas- sify the opinions expressed in various online forums or
platforms. This enables better understanding of the user's sentiment or
feeling towards

a particular subject matter \protect\hyperlink{_bookmark27}{{[}7{]}}.

The paper presents a system that not only recommends movies to the users
but also analyses and classifies the reviews into positive or negative.
The movie recommendation part is performed using Cosine Similarity and a
comparison is drawn between SVM and the NB algo- rithm to perform the
Sentiment Analysis of the reviews.

The objective of the study is to deal with the large volume of data and
filter useful information, recommend similar movies based on user's
choice and perform Sentimental Analysis on the reviews of the movie
chosen.

we are modifying in preprocessing in converting textual data into
numerical representations (TF-IDF vectors) and then calculate the
similarity between documents based on their TF-IDF representations using
cosine similarity. In addition to utilizing 10 classifiers, this paper
evaluates their performance individually. The results indicate that the
K-Nearest Neighbors (KNN) classifier achieved the lowest accuracy of
96.749\%, while the AdaBoostClassifier demonstrated the highest accuracy
of 99.061\%. Furthermore, combining different classifiers in pairs using
a voting ensemble approach revealed interesting insights. For instance,
the combination of KNN and SVC yielded the minimum accuracy of 97.471\%,
whereas the combination of AdaBoostClassifier and
GradientBoostingClassifier achieved the highest accuracy of 99.205\%.
Additionally, utilizing three classifiers, 'Decision Tree',
'ExtraTreesClassifier'and'AdaBoostClassifier',showcases remarkable
performance, with AdaBoostClassifier achieving a maximum accuracy of
99.422\%.so the results of what we do is better than the old one, then
we recommend to use our methodologies.

The paper follows the given structure;
\protect\hyperlink{related-works}{Section 2} covers the Literature
Review. \protect\hyperlink{methodology}{Section 3}, discusses the
Methodology which includes the dataset, the pre-processing of data,
mining of data for movie recommendation, machine learning for sentiment
analysis and finally, the performance report.
\protect\hyperlink{_bookmark10}{Section 4} comprises the results and
discussions. Section 5 contains pseudocode. Representing workflow in
\protect\hyperlink{flowchart}{Section 6} using flowchart. Finally, the
conclusion can be found in \protect\hyperlink{conclusion}{Section 7} And
References in \protect\hyperlink{references}{Section 8}.
\end{quote}

Fig. 1. Flowchart of proposed method.

\begin{quote}
\includegraphics[width=6.52966in,height=4.14394in]{media/image1.jpeg}
\end{quote}

\hypertarget{related-works}{%
\subsection{Related works}\label{related-works}}

\begin{quote}
In this section, the various existing methods and the drawbacks of the
existing work are discussed in detail.

In this paper, the authors propose a hybrid approach that combines a
content-based approach with genre correlation to implement a recom-
mendation system. This system takes into account both the user ratings
of the movie as well as their genres while making recommendations to the
user \protect\hyperlink{_bookmark24}{{[}4{]}}.

In this paper, a new similarity algorithm is introduced. This is called
User Profile Correlation-based Similarity (UPCSim) and it allows other
user behavioral data to influence the accuracy of the recommendation. It
calculates the weights of similarity which depend on the user's rating
and the user's behaviour value and classifies the preferences of the user
using K-nearest Neighbours algorithm. While this approach shows a de-
crease in the Mean Absolute Error (1.64\%) and the Root Mean Square
Error (1.4\%), it requires more computation time
\protect\hyperlink{_bookmark28}{{[}8{]}}.

The authors discuss the implementation of a movie recommendation system
using two algorithms, Cosine Similarity and K-Nearest Neigh- bours. The
movie recommendation is done using the cosine similarity algorithm. A
normalized popular score is used to obtain the function for computing
distance and the K-Nearest Neighbours algorithm is applied to enhance
the accuracy \protect\hyperlink{_bookmark29}{{[}9{]}}.

In this paper, a hybrid recommendation system is proposed that uses
sentiment analysis of user tweets for movies to obtain a sentiment score
to improve the recommendation made to the users using a weighted fusion
score method \protect\hyperlink{_bookmark30}{{[}10{]}}.

This paper implements five machine learning classifiers -- Multino- mial
Naïve Bayes, SVM, Decision Tree, Bernoulli Naïve Bayes, Maximum Entropy
are applied on the pre-processed data containing feature vectors to
classify the movie reviews data
\protect\hyperlink{_bookmark31}{{[}11{]}}.

Considering the utilization of 10 classifiers, this paper assesses their
individual accuracies. Notably, the K-Nearest Neighbors (KNN) classifier
achieves the lowest accuracy of 96.749\%, while the AdaBoostClassifier
demonstrates the highest accuracy of 99.061\%. Furthermore, through the
combination of various classifiers in pairs and employing a voting
ensemble approach, insights into performance metrics are gained. For
instance, the combination of KNN and SVC yields a minimum accuracy of
97.471\%, while the combination of AdaBoostClassifier and
GradientBoostingClassifier achieves the highest accuracy of 99.205\%.
\end{quote}

\hypertarget{methodology}{%
\subsection{Methodology}\label{methodology}}

\begin{quote}
Under this section, the methods used for the execution of the study and
implementation of the algorithms have been discussed. The diagram below
shows the flowchart of the methodology.

\protect\hyperlink{_bookmark3}{Fig. 1} explains the methodology that has
been used in the project. The study has used a dataset to train the
cosine similarity model which is used for recommending movies. Then
using another dataset, Naïve Bayes (NB) and Support Vector Machine (SVM)
Classifier for Sentiment

Analysis has been trained. Now a movie name is taken as an input and
sent to the movie recommendation model to predict similar movies.
Through web scraping from the IMDB site, the reviews of that movie are
obtained and sent to the Sentiment Analysis model for classifying the
reviews as positive or negative. \textbf{We mainly using 10 classifiers
`Bagging', `Random Forest', `Decision Tree', `Perceptron', `SVC',`KNN',
`Logistic Regression',}

\textbf{`ExtraTreesClassifier',`GradientBoostingClassifier'and
`AdaBoostClassifier'}

\textbf{And we combine two classifiers and three classifiers.}

1 Dataset

Three datasets have been used for study. 2 of them are for Movie
Recommendation and 1 is for Sentiment Analysis. The ones used for
recommendation are `tmdb\_5000\_movies.csv', `tbmd\_5000\_credits.csv'
and the one used for sentiment analysis is `reviews.t\textsc{x}t'. The 2
datasets used in movie recommendation are then merged to form a single
data set and the columns kept under it are `movie\_id', `title' and
`tags'.

The reviews data set has only 2 columns, one for the `reviews' and other
for the `comments'. The positive comments have been labelled as 1 and
the negative ones have been labelled as 0. There are 3943 positive
comments and 2975 negative comments .

1 Data pre- processing

After merging the 2 datasets into a single dataset, only the essential
columns such as `movie\_id', `title'. `overview', `genres', `keywords','
cast' and `crew' are kept, rest are removed from the dataset .

Then using Abstract Syntax trees, the columns of `genres', `keywords',
`cast', `crew' have been refined.

Furthermore, these columns have been combined under `tags'. Then using
the count vectorizer, the column `tags' is tokenised. Tokenising means
to divide the sentences into words. Here the pre-processing for Movie
Recommendation comes to an end.

The pre-processing for Sentiment Analysis requires Natural Language Tool
Kit (NLTK). The NLTK is a leading platform for building Python programs
to work with human language data.

This is a standard python library used for natural language process- ing
and computational linguistics. Using this library, the stop words are
downloaded. Stop Words are the usually used words in any language.
E\textsc{x}amples of such words include `a', `an', `the', `if', `or'.
They are used in Text Mining and Natural Language Processing (NLP) to
eliminate such words as they carry very little useful information. Then
using the Tfid- fVectoriser, the column of Comments is tokenised.

Now the Data Pre-processing of the datasets ends here.

The \protect\hyperlink{_bookmark6}{Fig. 2} shows the dataset used for
Movie Recommendation, which has the `movie\_id', `title' and `tags'.

The \protect\hyperlink{_bookmark7}{Fig. 3} shows the dataset for
sentiment analysis, has 2 columns, one for the `comments' and other for
the `reviews'.

1 Data mining for movie recommendation

\includegraphics[width=4.66667in,height=1.19714in]{media/image2.png}\protect\hypertarget{_bookmark6}{}{}\textbf{Fig.
2.} Final dataset being used for movie recommen- dation.

\includegraphics[width=4.66667in,height=2.05714in]{media/image3.png}\protect\hypertarget{_bookmark7}{}{}\textbf{Fig.
3.} Dataset for sentiment analysis.

Using the sklearn library in Python, the Cosine Similarity algorithm is
used.

After the user is prompted to enter a movie, the algorithm provides 5
other movies like the one used as an input by the user.

In cosine similarity, vectors are taken as the data objects in data
sets, when defined in a product space, the similarity is figured out. The
smaller this distance, the higher the similarity, but the larger the
dis- tance, the lower the similarity. Cosine similarity is a measure
that helps

\protect\hyperlink{_bookmark11}{Eq. (2}) calculates function for two
points X and X' computes the similarity or how close they are to each
other.

`𝜎' is the variance and the hyperparameter. \textbar\textbar X --
X'\textbar\textbar{} is the Euclidean (L\textsubscript{2}-norm) Distance
between two points X and X'.

The NB Algorithm uses conditional probability to classify the given data
set. Bayes theorem is used for the computation and used class levels
represented as feature values or vectors of predictors for classification
\protect\hyperlink{_bookmark32}{{[}12{]}}.

\protect\hypertarget{_bookmark8}{}{}to find out how similar data objects
are, regardless of size. Mathemat- ically, it is the cosine of the angle
between two vectors projected in a

𝑃 (𝐴\protect\hyperlink{_bookmark8}{\textbar{}}𝐵) = 𝑃 (𝐵\textbar 𝐴)𝑃 (𝐴)

(3)

multi-dimensional space \protect\hyperlink{_bookmark29}{{[}9{]}}.

\protect\hyperlink{_bookmark8}{Eq. (3}) calculates the conditional
probability of event A such that

\protect\hypertarget{_bookmark9}{}{}𝐶𝑜𝑠 𝜃 =

\uline{𝑎⃖⃗, 𝑏⃖⃗} =

‖ ‖

∑𝑛 𝑎\textsubscript{𝑖}𝑏\textsubscript{𝑖}

√

(1)

B has already occurred, and this is used to for calculation in the NB
\end{quote}

𝑎⃖⃗ 𝑏⃖⃗

\begin{quote}
∑𝑛 𝑎2 ∑𝑛 𝑏2

classifier.

The angle between two vectors determines its direction and is mea-

sured in `𝜃'. This angle 𝜃 can be calculated by using
\protect\hyperlink{_bookmark9}{Eq. (1)}.

When 𝜃 = 0 °, the `\textsc{x}` and` y` vectors overlap and prove to be
similar. When 𝜃 = 90 °, the `\textsc{x}` and` y` vectors are therefore
dissimilar.

1 Machine learning for sentiment analysis

As discussed earlier, 2 algorithms have been used, SVC and NB. The data
set is split into testing set and training set, the testing set is

0.20 and training size is 0.80.

After this the 2 models are fitted. To increase the accuracy of both the
models, hyperparameter tuning is applied on both the models.

SVC is a supervised algorithm in the machine learning domain used
\protect\hypertarget{_bookmark10}{}{}for both classification and
regression. It classifies the info points by find- ing a hyperplane in an
N-dimensional space \protect\hyperlink{_bookmark32}{{[}12{]}}. The
hyperplane is simply a line if the amount of input features is 2,
however it's 2-D hy- perplane if the amount of input features is three.

Radial Basis Function Kernel was used in the model which is a type of
Non-linear SVM \protect\hyperlink{_bookmark33}{{[}13{]}}.

It is referred to as RBF kernel. Metric squared Euclidean distance is
used for distance. It is used to draw completely non-linear hyperplanes.
\end{quote}

\begin{enumerate}
\def\labelenumi{\arabic{enumi}.}
\item
  \begin{quote}
  Gaussian,
  \end{quote}
\item
  \begin{quote}
  Multinomial,
  \end{quote}
\item
  \begin{quote}
  Bernoulli.
  \end{quote}
\end{enumerate}

\begin{quote}
The proposed system uses the multinomial NB model, which pre- dicts the
badge of a text such as a piece of email or newspaper article. The
probability of each badge is calculated for a given sample and then
badge with the highest probability is given as output.

Since this algorithm is mainly used for natural language processing and
text data analysis it was a perfect choice for sentiment analysis of
movie reviews
\end{quote}

\hypertarget{results}{%
\subsection{Results}\label{results}}

\begin{quote}
Using the Cosine Similarity algorithm, prediction of movies like the
ones input by the user can be made.

The \protect\hyperlink{_bookmark12}{Fig. 4} shows the Cosine Similarity
matri\textsc{x} of the values in the dataset.

The \protect\hyperlink{_bookmark12}{Fig. 5} shows that a negative
comment has been classified as 0 by the NB classifier.

The \protect\hyperlink{_bookmark12}{Fig. 6} shows that a negative
comment is classified as 0 by the

\protect\hypertarget{_bookmark11}{}{}𝐾(𝑋, 𝑋\textsuperscript{′})

= 𝑒𝑥𝑝(−

𝑋 − 𝑋′

2𝜎2

(2)

SVC.

An integral part of the study was comparison between the 2 algo- rithms
as well.

\includegraphics[width=4.66788in,height=1.82333in]{media/image4.png}
\end{quote}

\includegraphics[width=4.67287in,height=1.43333in]{media/image5.jpeg}\includegraphics[width=4.67287in,height=1.43333in]{media/image6.jpeg}

\begin{quote}
\protect\hypertarget{_bookmark12}{}{}\textbf{Table 1}

Comparison of models

\textbf{Fig. 4.} Value of array after fitting cosine similarity on the
dataset.

\textbf{Fig. 5.} Output for NB algorithm.

\textbf{Fig. 6.} Output by SVC.

SVM (Proposed) 0.9863 0.98278 0.9937 0.984

Bernoulli's Naive Bayes(E\textsc{x}isting)
\protect\hyperlink{_bookmark31}{{[}11{]}} 0.875 0.884 0.8633 0.8735

Multinomial NB(E\textsc{x}isting)
\protect\hyperlink{_bookmark31}{{[}11{]}} 0.885 0.9294 0.8333 0.8787

SVM(E\textsc{x}isting) \protect\hyperlink{_bookmark31}{{[}11{]}} 0.8733
0.859 0.8933 0.8753

NB(E\textsc{x}isting) \protect\hyperlink{_bookmark25}{{[}5{]}} 0.8183
0.84 0.79 0.82

SVM(E\textsc{x}isting) \protect\hyperlink{_bookmark25}{{[}5{]}} 0.8745
0.87 0.88 0.88

Random Forest (E\textsc{x}isting)
\protect\hyperlink{_bookmark25}{{[}5{]}} 0.9601 0.93 1.00 0.96

Stacked-LSTM(E\textsc{x}isting)
\protect\hyperlink{_bookmark34}{{[}14{]}} 0.9365 0.94 0.94 -

Minimal-RNN(E\textsc{x}isting) \protect\hyperlink{_bookmark34}{{[}14{]}}
0.8564 0.86 0.86 -

CNN(E\textsc{x}isting) \protect\hyperlink{_bookmark35}{{[}15{]}} 0.8915
0.8259 0.8246 0.8253

LSTM(E\textsc{x}isting) \protect\hyperlink{_bookmark35}{{[}15{]}} 0.9550
0.9087 0.8228 0.8636

The \protect\hyperlink{_bookmark12}{Table 1} shows the comparison
between different Accuracy, Pre- cision, Recall and AUC scores of the 2
proposed models and 3 existing models (Bernoulli's Naïve Bayes,
Multinomial NB, SVM) studied in
\protect\hyperlink{_bookmark31}{{[}11{]}}, 3 existing models (SVM, NB,
Random Forest) studied in \protect\hyperlink{_bookmark25}{{[}5{]}} 2
existing models (Stacked-LSTM, Minimal-RNN) studied in
\protect\hyperlink{_bookmark34}{{[}14{]}} and 2 existing models (CNN,
LSTM) studied in \protect\hyperlink{_bookmark35}{{[}15{]}}.

Proposed SVM model is better than NB in all parameters. The
\protect\hyperlink{_bookmark16}{Fig. 7} shows the ROC Curves between the
2 algorithms.

One of the examples that taken for the study is the movie Spectre.

Through the Cosine Similarity algorithm, predictions of 5 other movies -
Quantum of Solace, Never Say Never Again, Skyfall, Thunder- ball, From
Russia with Love were made.

\protect\hyperlink{_bookmark17}{Fig. 8}, shows the reviews about the
movie - Spectre, entered by the user. Web scraping is used to get the
taglines of the reviews. Web scrap- ing is performed from the IMDB
website and perform Sentiment Analysis on it using the NB and SVC
algorithm.

After performing Sentiment Analysis,
(\protect\hyperlink{_bookmark18}{Figs. 9},
\protect\hyperlink{_bookmark19}{11}), one can tell whether it is a good
movie or not. Algorithm 1: Clear the accuracy results obtained by using
10 classifiers and select the maximum accuracy.

Algorithm 2: Determine the maximum accuracy achieved by combining two
classifiers. Algorithm 3: Identify the maximum accuracy obtained by
combining three classifiers.

So, the review `Enjoyable installment in Bond series with lots of noisy
action, thrills, emotion and spectacular scenes' gets assigned a value 1
which is the interpretation of a good review to the movie.

Another review `the James Bond franchise should have ended decades ago'
gets assigned a value 0 which is the interpretation of a bad review to
the movie.
\end{quote}

Comparison between the SVC and Naive Bayes (NB) algorithms in terms of
accuracy shows that the accuracy of SVC is higher in all two figures
(1,2).

1:\includegraphics[width=4.10897in,height=2.24153in]{media/image7.png}

2:

\includegraphics[width=3.96795in,height=2.49722in]{media/image8.png}

\begin{quote}
\includegraphics[width=4.66667in,height=3.21143in]{media/image9.jpeg}\protect\hypertarget{_bookmark16}{}{}\textbf{Fig.
7.} ROC Curves.

\includegraphics[width=4.66667in,height=2.33429in]{media/image10.jpeg}\protect\hypertarget{_bookmark17}{}{}\textbf{Fig.
8.} Prediction of movies using the cosine similarity algorithm.

\includegraphics[width=4.66667in,height=2.29714in]{media/image11.png}\protect\hypertarget{_bookmark18}{}{}\textbf{Fig.
9.} Sentiment analysis on reviews using NB algo- rithm.
\end{quote}

\hypertarget{visualizing-confusion}{%
\section{\texorpdfstring{\textbf{Visualizing
Confusion}}{Visualizing Confusion}}\label{visualizing-confusion}}

\includegraphics[width=6.22917in,height=4.31537in]{media/image12.png}

\begin{quote}
\includegraphics[width=4.66667in,height=2.28857in]{media/image13.png}\protect\hypertarget{_bookmark19}{}{}\textbf{Fig.
11.} Sentiment analysis on reviews using SVC.
\end{quote}

\textbf{Fig12: Results before the modification:}

\includegraphics[width=3.29167in,height=2in]{media/image14.png}

\textbf{Fig13: Results after the modification:}

\includegraphics[width=3.99167in,height=2.05in]{media/image15.png}

Algorithm 1: Apply ten classifiers.

\begin{quote}
\includegraphics[width=6.32639in,height=3.64583in]{media/image16.png}
\end{quote}

Algorithm 2: combine two classifiers.

\includegraphics[width=7.39583in,height=0.55in]{media/image17.png}

Algorithm 3: combine three classifiers.

\includegraphics[width=7.39583in,height=0.50903in]{media/image18.png}

\hypertarget{pseudocode}{%
\subsection{Pseudocode}\label{pseudocode}}

IMPORT NECESSARY LIBRARIES

LOAD MOVIE DATA FROM CSV FILES

MERGE TWO DATASETS BASED ON TITLE COLUMN

SELECT ESSENTIAL COLUMNS AND DROP UNNECESSARY ONES

DEFINE FUNCTION TO EXTRACT NAMES FROM JSON OBJECTS

APPLY THIS FUNCTION TO GENRES, KEYWORDS, AND CAST COLUMNS

DROP NA VALUES IN THE DATASET

DEFINE FUNCTION TO FETCH DIRECTOR NAME FROM CREW COLUMN

APPLY THIS FUNCTION TO CREW COLUMN

PROCESS OVERVIEW COLUMN BY SPLITTING INTO INDIVIDUAL WORDS

COMBINE ALL COLUMNS INTO A SINGLE TAGS COLUMN

CREATE NEW DATASET WITH ONLY ESSENTIAL COLUMNS

TRANSFORM TAGS COLUMN INTO VECTOR FORMAT USING TF-IDF VECTORIZER

DEFINE RECOMMENDATION SYSTEM FUNCTION

CALCULATE SIMILARITY BETWEEN MOVIES USING COSINE SIMILARITY

RETURN TOP 5 RECOMMENDED MOVIES FOR GIVEN INPUT MOVIE

LOAD REVIEWS DATA FROM TEXT FILE

PREPROCESS REVIEWS DATA BY APPLYING TF-IDF VECTORIZER

PERFORM GRID SEARCH TO FIND BEST PARAMETERS FOR XGBOOST CLASSIFIER

TRAIN XGBOOST CLASSIFIER ON PREPROCESSED REVIEWS DATA

EVALUATE MODEL PERFORMANCE USING ACCURACY, F1 SCORE, PRECISION, AND
RECALL

DRAW ROC CURVE AND CONFUSION MATRIX FOR MODEL EVALUATION

DEFINE PIPELINE FOR GRID SEARCH AND MODEL TRAINING

PERFORM GRID SEARCH TO FIND BEST PARAMETERS FOR SVC CLASSIFIER

TRAIN SVC CLASSIFIER ON PREPROCESSED REVIEWS DATA

EVALUATE MODEL PERFORMANCE USING ACCURACY, F1 SCORE, PRECISION, AND
RECALL

DRAW ROC CURVE AND CONFUSION MATRIX FOR MODEL EVALUATION

COMPARE PERFORMANCE OF XGBOOST AND SVC CLASSIFIERS USING
CROSS-VALIDATION

DRAW BOX PLOT TO COMPARE ALGORITHMS

DEFINE FUNCTION TO GET IMDB REVIEWS FOR GIVEN MOVIE TITLE

FETCH IMDB REVIEWS USING IMDbPY LIBRARY

PREPROCESS REVIEWS DATA BY APPLYING TF-IDF VECTORIZER

PREDICT REVIEW SENTIMENT USING TRAINED MODELS

PRINT RESULTS

\begin{enumerate}
\def\labelenumi{\arabic{enumi}.}
\setcounter{enumi}{5}
\item ~
  \hypertarget{flowchart}{%
  \subsection{Flowchart}\label{flowchart}}
\item ~
  \hypertarget{conclusion}{%
  \subsection{Conclusion}\label{conclusion}}
\end{enumerate}

\hypertarget{this-paper-presents-a-comprehensive-analysis-of-two-key-components-the-movie-recommendation-system-and-sentiment-analysis.-through-rigorous-testing-and-evaluation-important-insights-and-conclusions-have-been-drawn.-the-movie-recommendation-system-utilizes-the-cosine-similarity-algorithm-to-suggest-relevant-movies-based-on-various-factors-such-as-genre-overview-cast-and-ratings.-despite-multiple-tests-cosine-similarity-consistently-delivers-accurate-movie-recommendations-showcasing-its-reliability.-in-the-realm-of-sentiment-analysis-the-study-employs-two-algorithms-nauxefve-bayes-nb-and-support-vector-machine-svc-to-classify-reviews-as-positive-or-negative.-given-the-diverse-nature-of-reviews-determining-the-optimal-algorithm-for-classification-is-crucial.-experimental-results-reveal-that-svc-demonstrates-slightly-better-accuracy-compared-to-nb.-several-avenues-for-future-research-are-identified-enhancing-the-accuracy-of-sentiment-analysis-to-effectively-classify-sarcastic-or-ironic-reviews.-extending-sentiment-analysis-to-reviews-in-languages-other-than-english.-tailoring-movie-recommendations-based-on-users-preferences-including-cast-genre-and-release-year.-despite-its-accuracy-the-system-has-limitations.-failure-to-recommend-movies-occurs-when-the-users-input-does-not-match-the-dataset-format.-additionally-linguistic-barriers-restrict-sentiment-analysis-to-english-reviews-only.-moreover-misclassification-may-occur-in-the-presence-of-sarcastic-or-ironic-reviews.-furthermore-the-study-explores-the-utilization-of-10-classifiers-with-k-nearest-neighbors-knn-achieving-the-lowest-accuracy-of-96.749-while-adaboostclassifier-demonstrates-the-highest-accuracy-of-99.061.-employing-voting-ensembles-with-paired-classifiers-reveals-insightful-performance-metrics-with-the-combination-of-knn-and-svc-yielding-a-minimum-accuracy-of-97.471-and-adaboostclassifier-and-gradientboostingclassifier-achieving-the-highest-accuracy-of-99.205.-additionally-utilizing-three-classifiers-decision-tree-extratreesclassifier-and-adaboostclassifier-showcases-remarkable-performance-with-adaboostclassifier-achieving-a-maximum-accuracy-of-99.422.-these-findings-underscore-the-importance-of-recommending-the-highest-accuracy-classifiers-for-user-utilization-enhancing-overall-system-efficiency.}{%
\subsection{\texorpdfstring{\textbf{This paper presents a comprehensive
analysis of two key components: the Movie Recommendation System and
Sentiment Analysis. Through rigorous testing and evaluation, important
insights and conclusions have been drawn. The Movie Recommendation
System utilizes the Cosine Similarity algorithm to suggest relevant
movies based on various factors such as genre, overview, cast, and
ratings. Despite multiple tests, Cosine Similarity consistently delivers
accurate movie recommendations, showcasing its reliability. In the realm
of Sentiment Analysis, the study employs two algorithms, Naïve Bayes
(NB) and Support Vector Machine (SVC), to classify reviews as positive
or negative. Given the diverse nature of reviews, determining the
optimal algorithm for classification is crucial. Experimental results
reveal that SVC demonstrates slightly better accuracy compared to NB.
Several avenues for future research are identified: Enhancing the
accuracy of Sentiment Analysis to effectively classify sarcastic or
ironic reviews. Extending Sentiment Analysis to reviews in languages
other than English. Tailoring movie recommendations based on users'
preferences, including cast, genre, and release year. Despite its
accuracy, the system has limitations. Failure to recommend movies occurs
when the user's input does not match the dataset format. Additionally,
linguistic barriers restrict Sentiment Analysis to English reviews only.
Moreover, misclassification may occur in the presence of sarcastic or
ironic reviews. Furthermore, the study explores the utilization of 10
classifiers, with K-Nearest Neighbors (KNN) achieving the lowest
accuracy of 96.749\%, while AdaBoostClassifier demonstrates the highest
accuracy of 99.061\%. Employing voting ensembles with paired classifiers
reveals insightful performance metrics, with the combination of KNN and
SVC yielding a minimum accuracy of 97.471\%, and AdaBoostClassifier and
GradientBoostingClassifier achieving the highest accuracy of 99.205\%.
Additionally, utilizing three classifiers, 'Decision Tree',
'ExtraTreesClassifier', and 'AdaBoostClassifier', showcases remarkable
performance, with AdaBoostClassifier achieving a maximum accuracy of
99.422\%. These findings underscore the importance of recommending the
highest accuracy classifiers for user utilization, enhancing overall
system efficiency.}
}{This paper presents a comprehensive analysis of two key components: the Movie Recommendation System and Sentiment Analysis. Through rigorous testing and evaluation, important insights and conclusions have been drawn. The Movie Recommendation System utilizes the Cosine Similarity algorithm to suggest relevant movies based on various factors such as genre, overview, cast, and ratings. Despite multiple tests, Cosine Similarity consistently delivers accurate movie recommendations, showcasing its reliability. In the realm of Sentiment Analysis, the study employs two algorithms, Naïve Bayes (NB) and Support Vector Machine (SVC), to classify reviews as positive or negative. Given the diverse nature of reviews, determining the optimal algorithm for classification is crucial. Experimental results reveal that SVC demonstrates slightly better accuracy compared to NB. Several avenues for future research are identified: Enhancing the accuracy of Sentiment Analysis to effectively classify sarcastic or ironic reviews. Extending Sentiment Analysis to reviews in languages other than English. Tailoring movie recommendations based on users' preferences, including cast, genre, and release year. Despite its accuracy, the system has limitations. Failure to recommend movies occurs when the user's input does not match the dataset format. Additionally, linguistic barriers restrict Sentiment Analysis to English reviews only. Moreover, misclassification may occur in the presence of sarcastic or ironic reviews. Furthermore, the study explores the utilization of 10 classifiers, with K-Nearest Neighbors (KNN) achieving the lowest accuracy of 96.749\%, while AdaBoostClassifier demonstrates the highest accuracy of 99.061\%. Employing voting ensembles with paired classifiers reveals insightful performance metrics, with the combination of KNN and SVC yielding a minimum accuracy of 97.471\%, and AdaBoostClassifier and GradientBoostingClassifier achieving the highest accuracy of 99.205\%. Additionally, utilizing three classifiers, 'Decision Tree', 'ExtraTreesClassifier', and 'AdaBoostClassifier', showcases remarkable performance, with AdaBoostClassifier achieving a maximum accuracy of 99.422\%. These findings underscore the importance of recommending the highest accuracy classifiers for user utilization, enhancing overall system efficiency. }}\label{this-paper-presents-a-comprehensive-analysis-of-two-key-components-the-movie-recommendation-system-and-sentiment-analysis.-through-rigorous-testing-and-evaluation-important-insights-and-conclusions-have-been-drawn.-the-movie-recommendation-system-utilizes-the-cosine-similarity-algorithm-to-suggest-relevant-movies-based-on-various-factors-such-as-genre-overview-cast-and-ratings.-despite-multiple-tests-cosine-similarity-consistently-delivers-accurate-movie-recommendations-showcasing-its-reliability.-in-the-realm-of-sentiment-analysis-the-study-employs-two-algorithms-nauxefve-bayes-nb-and-support-vector-machine-svc-to-classify-reviews-as-positive-or-negative.-given-the-diverse-nature-of-reviews-determining-the-optimal-algorithm-for-classification-is-crucial.-experimental-results-reveal-that-svc-demonstrates-slightly-better-accuracy-compared-to-nb.-several-avenues-for-future-research-are-identified-enhancing-the-accuracy-of-sentiment-analysis-to-effectively-classify-sarcastic-or-ironic-reviews.-extending-sentiment-analysis-to-reviews-in-languages-other-than-english.-tailoring-movie-recommendations-based-on-users-preferences-including-cast-genre-and-release-year.-despite-its-accuracy-the-system-has-limitations.-failure-to-recommend-movies-occurs-when-the-users-input-does-not-match-the-dataset-format.-additionally-linguistic-barriers-restrict-sentiment-analysis-to-english-reviews-only.-moreover-misclassification-may-occur-in-the-presence-of-sarcastic-or-ironic-reviews.-furthermore-the-study-explores-the-utilization-of-10-classifiers-with-k-nearest-neighbors-knn-achieving-the-lowest-accuracy-of-96.749-while-adaboostclassifier-demonstrates-the-highest-accuracy-of-99.061.-employing-voting-ensembles-with-paired-classifiers-reveals-insightful-performance-metrics-with-the-combination-of-knn-and-svc-yielding-a-minimum-accuracy-of-97.471-and-adaboostclassifier-and-gradientboostingclassifier-achieving-the-highest-accuracy-of-99.205.-additionally-utilizing-three-classifiers-decision-tree-extratreesclassifier-and-adaboostclassifier-showcases-remarkable-performance-with-adaboostclassifier-achieving-a-maximum-accuracy-of-99.422.-these-findings-underscore-the-importance-of-recommending-the-highest-accuracy-classifiers-for-user-utilization-enhancing-overall-system-efficiency.}}

\hypertarget{references}{%
\subsection{References}\label{references}}

\begin{enumerate}
\def\labelenumi{\arabic{enumi}.}
\item
  \href{http://refhub.elsevier.com/S2666-285X(22)00017-6/sbref0001}{N.
  Nassar, A. Jafar, Y. Rahhal, A novel deep multi-criteria collaborative
  filtering model for recommendation system, Knowl. Based Syst. 187
  (2020) 104811.}
\item
  \href{http://refhub.elsevier.com/S2666-285X(22)00017-6/sbref0003}{A.
  Beheshti, S. Yakhchi, S. Mousaeirad, S.M. Ghafari, S.R. Goluguri, M.A.
  Edrisi, Towards cognitive recommender systems, Algorithms 13 (8)
  (2020) 176.}
\item
  \href{http://refhub.elsevier.com/S2666-285X(22)00017-6/sbref0005}{S.
  Sharma, V. Rana, M. Malhotra, Automatic recommendation system based on
  hy- brid filtering algorithm, Educ. Inf. Technol. 27 (2021) 1--16.}
\item
  \href{http://refhub.elsevier.com/S2666-285X(22)00017-6/sbref0007}{S.R.S.
  Reddy, S. Nalluri, S. Kunisetti, S. Ashok, B. Venkatesh, Content-based
  movie recommendation system using genre correlation, in: Smart
  Intelligent Computing and Applications, Springer, Singapore, 2019, pp.
  391--397.}
\item
  M. Yasen, S. Tedmori, Movies reviews sentiment analysis and
  classification, in: Proceedings of the IEEE Jordan International Joint
  Conference on Elec- trical Engineering and Information Technology,
  JEEIT, 2019, pp. 860--865,
  doi:\href{https://doi.org/10.1109/JEEIT.2019.8717422}{10.1109/JEEIT.2019.8717422}.
\item
  \href{http://refhub.elsevier.com/S2666-285X(22)00017-6/sbref0011}{N.
  Rajput, S. Chauhan, Analysis of various sentiment analysis techniques,
  Int. J. Comput. Sci. Mob. Comput. 8 (2) (2019) 75--79.}
\item
  \href{http://refhub.elsevier.com/S2666-285X(22)00017-6/sbref0013}{Z.
  Shaukat, A.A. Zulfiqar, C. Xiao, M. Azeem, T. Mahmood, Sentiment
  analysis on IMDB using lexicon and neural networks, SN Appl. Sci. 2
  (2) (2020) 1--10.}
\item
  \href{http://refhub.elsevier.com/S2666-285X(22)00017-6/sbref0015}{T.
  Widiyaningtyas, I. Hidayah, T.B. Adji, User profile correlation-based
  similarity (UPCSim) algorithm in movie recommendation system, J. Big
  Data 8 (2021) 52.}
\item
  \href{http://refhub.elsevier.com/S2666-285X(22)00017-6/sbref0017}{R.H.
  Singh, S. Maurya, T. Tripathi, T. Narula, G. Srivastav, Movie
  recommendation system using cosine similarity and KNN, Int. J. Eng.
  Adv. Technol. (IJEAT) 9 (5) (2020) 2--3 ISSN: 2249
  --8958VolumeIssueJune.}
\item
  \href{http://refhub.elsevier.com/S2666-285X(22)00017-6/sbref0019}{S.
  Kumar, K. De, P.P. Roy, Movie recommendation system using sentiment
  analysis from microblogging data, IEEE Trans. Comput. Soc. Syst. 7 (4)
  (2020) 915--923.}
\item
  \href{http://refhub.elsevier.com/S2666-285X(22)00017-6/sbref0020}{A.
  Rahman, M.S. Hossen, Sentiment analysis on movie review data using
  machine learning approach, in: Proceedings of the International
  Conference on Bangla Speech and Language Processing (ICBSLP), IEEE,
  2019, pp. 1--4.}
\item
  \href{http://refhub.elsevier.com/S2666-285X(22)00017-6/sbref0021}{S.
  Uddin, A. Khan, M.E. Hossain, M.A. Moni, Comparing different supervised
  ma- chine learning algorithms for disease prediction, BMC Med. Inf.
  Decis. Mak. 19 (1) (2019) 1--16.}
\item
  \href{http://refhub.elsevier.com/S2666-285X(22)00017-6/sbref0022}{S.
  Ghosh, A. Dasgupta, A. Swetapadma, A study on support vector machine
  based linear and non-linear pattern classification, in: Proceedings of
  the International Con- ference on Intelligent Sustainable Systems
  (ICISS), IEEE, 2019, pp. 24--28.}
\item
  \href{http://refhub.elsevier.com/S2666-285X(22)00017-6/sbref0023}{K.
  Dashtipour, M. Gogate, A. Adeel, H. Larijani, A. Hussain, Sentiment
  analysis of Persian movie reviews using deep learning, Entropy 23 (5)
  (2021) 596.}
\item
  \href{http://refhub.elsevier.com/S2666-285X(22)00017-6/sbref0024}{S.
  Soubraylu, R. Rajalakshmi, Hybrid convolutional bidirectional
  recurrent neural network based sentiment analysis on movie reviews,
  Comput. Intell. 37 (2) (2021) 735--757.}
\end{enumerate}

\end{document}
